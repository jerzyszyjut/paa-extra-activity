\documentclass{article}
\usepackage[T1]{fontenc}
\usepackage[polish]{babel}
\usepackage[utf8]{inputenc}
\usepackage{amssymb}
\usepackage{pythonhighlight}
\usepackage{tikz}
\usepackage{amsmath}
\usepackage[a4paper, left=40mm, right=40mm, top=30mm, bottom=30mm]{geometry}
\usepackage[indent=0pt]{parskip}
\usepackage[backend=biber]{biblatex}
\addbibresource{bibliography.bib}

\title{Zadanie "Proof" 1}
\author{Jerzy Szyjut, nr albumu: 193064}
\date{21.01.2023r.}

\usepackage{csquotes}
\begin{document}

\maketitle

\begin{section}
 {Treść zadania}
 Przez $\sum$ będziemy rozumieli skończony alfabet złożony z liter (dowolnych symboli), np. $\sum = \{a, b, c\}$ lub
 $\sum = \{0, 1\}$. Przez \textit{słowo} nad alfabetem $\sum$ będziemy rozumieli dowolny skończony ciąg elementów z alfabetu,
 np. $aaba$ lub $001100$. Zbiór wszystkich skończonych słów nad alfabetem oznaczymy przez $\sum^+$. Wprowadzimy
 tzw. \textit{słowo puste} oznaczane przez $\epsilon \  (\epsilon \notin \sum)$ oraz oznaczenie $\sum^* = \sum^+ \cup \{\epsilon\}$.
 Dla dwóch słów $u, v \in \sum^*$ przez $uv$ będziemy rozumieli słowo powstałe przez \textit{konkatenację} (sklejenie) słów $u$ oraz $v$.
 Zdefinujemy słowo $\epsilon$ przez własności $\epsilon w = w\epsilon = w$, dla dowolnego $w \in \sum^*$. Długością słowa $w$,
 ozn. $|w|$, będziemy nazywali ilość kolejnych liter w słowie, np. $|aabab| = 5$. Niech $u, w \in \sum^*$, $u = u_1 ... u_k$
 oraz $w = w_1 ... w_n$, gdzie $k = |u|$ oraz $n = |w|$. Słowo $u$ nazwiemy \textit{podciągiem} słowa $w$ jeżeli $u_i = w_{j_i}$ dla $i \in \{1, ..., k\}$,
 gdzie $1 \leq j_1 < j_2 < ... < j_k \leq n$, np. $abbc$ jest podciągiem $xyababac$. Wprowadzimy oznaczenie $u \sqsubset w$, o ile $u$ jest
 podsłowem $w$.

 Zdefiniujemy problem znajdowania najdłuższych wspólnych podciągów dla danego zbioru słów. Niech
 $u, w \in \sum^*$, przez $nwp(u, w)$ będziemy rozumieli $max\{|v|: v \sqsubset u \land v \sqsubset w\}$, czyli długość najdłuższego wspólnego
 podciągu słów $u$ oraz $w$. Analogicznie możemy zdefiniować $nwp(w_1, ..., w_k)$ dla zbioru słów $\{w_1, ..., w_k\}$.

 Niech $L \subset \sum^+$. Problem $NWP$ zdefiniujemy jako problem znalezienia najdłuższego wspólnego podciągu
 słów ze zbioru $L$ (wersja optymalizacyjna). Wersja decyzyjna ma na wejściu dodatkowo liczbę $d \geq 0$ i
 pytamy czy $nwp(L) \geq d$. Wprowadzimy teraz wersję sparametryzowaną liczbami całkowitymi $p, q, r \geq 0$.
 Przez $NWP(p,q,r)$ będziemy rozumieli problem znajdowania najdłuższego wspólnego podciągu wszystkich
 słów ze zbioru $L \subset \sum^+$, gdzie $|L| \leq p$, $|\sum| \leq q$ oraz $r$ oznacza ograniczenie na długość zwartych jednoliterowych
 ciągów w słowach ze zbioru $L$. Np. zapisując słowo $aabbabcccdaa$ w postaci skompresowanej z krotnościami
 $a^2b^2abc^3da^2$ najdłuższy ciąg jednoliterowy to $c^3$ i ciąg ten należy do instancji dla $r = 3$, ale nie dla $r = 2$.
 W oznaczeniach przyjmiemy symbol "$\cdot$" oznaczający brak wybranej restrykcji.
\end{section}
\newpage
\begin{section}
 {$NWP(2, \cdot, \cdot) \in P$}
 Algorytm wielomianowy dla problemu $NWP(2, \cdot, \cdot)$ w języku $Python$ \cite{cormen}
 \begin{python}
def nwp(u, w):
    n = len(u)
    k = len(w)

    macierz = [[0] * (k + 1) for _ in range(n + 1)]

    for i in range(n + 1):
        macierz[i][0] = 0
    for j in range(k + 1):
        macierz[0][j] = 0

    for i in range(1, n + 1):
        for j in range(1, k + 1):
            if u[i - 1] == w[j - 1]:
                macierz[i][j] = macierz[i - 1][j - 1] + 1  
            else:
                macierz[i][j] = max(macierz[i - 1][j], macierz[i][j - 1])

    podciag = []
    i, j = n, k
    while i > 0 and j > 0:
        if u[i - 1] == w[j - 1]:
            podciag.append(u[i - 1])
            i -= 1
            j -= 1
        elif macierz[i - 1][j] > macierz[i][j - 1]:
            i -= 1
        else:
            j -= 1
            
    return "".join(reversed(podciag))
 \end{python}

 Powyższy algorytm jest implementacją algorytmu dynamicznego dla problemu \newline $NWP(2, \cdot, \cdot)$. 
Generuje on tablicę dwuwymiarową, gdzie w wierszach znajdują się litery słowa $u$, a w kolumnach litery słowa $w$.
W komórkach tablicy znajdują się długości najdłuższych wspólnych podciągów słów $u$ oraz $w$. 
Algorytm przechodzi po tablicy od lewej do prawej, od góry do dołu.
Jeżeli litery w słowach $u$ oraz $w$ są takie same, to w komórce znajdującej się w tym samym wierszu i kolumnie co litery,
które są takie same, zwiększamy wartość o 1. Jeżeli litery są różne, to w komórce znajdującej się w tym samym wierszu i kolumnie co litery,
które są różne, wpisujemy wartość większą z dwóch wartości znajdujących się w komórkach powyżej i po lewej stronie od komórki, w której się znajdujemy.
Po przejściu całej tablicy, w ostatniej komórce znajduje się długość najdłuższego wspólnego podciągu słów $u$ oraz $w$.
\end{section}
\newpage
\begin{section}
{Status problemu $NWP(\cdot, 2, 1)$}
Algorytm dla problemu $NWP(\cdot, 2, 1)$ w języku $Python$ wykonujący $\alpha$-redukcję do problemu $NWP(2, \cdot, \cdot)$
\begin{python}
def nwp_2_1(*sekwencje):
    dlugosc_najkrotszej_sekwencji = len(sekwencje[0])
    for sekwencja in sekwencje:
        if len(sekwencja) < dlugosc_najkrotszej_sekwencji:
            dlugosc_najkrotszej_sekwencji = sekwencja
    
    if dlugosc_najkrotszej_sekwencji == 0:
        return ""
            
    najkrotsze_sekwencje = set()
    for sekwencja in sekwencje:
        if len(sekwencja) == dlugosc_najkrotszej_sekwencji:
            najkrotsze_sekwencje.add(sekwencja)
            
    if len(najkrotsze_sekwencje) == 1:
        return najkrotsze_sekwencje.pop()

    if len(najkrotsze_sekwencje) == 2:
        return nwp(*najkrotsze_sekwencje)

\end{python}
 Zakładając alfabet $\sum = \{0, 1\}, |\sum| = 2 = q$ i $r=1$ każde słowo musi być ciągiem zaczynającym się od $0$ lub $1$
 i nie może zawierać dwóch takich samych liter obok siebie. Niech $n$ oznacza długość najkrótszego słowa ze zbioru $L$, 
 $min\{|w|: w \in L\}$. $NWP(\cdot, 2, 1) = n$ kiedy najkrótsze słowo jest jedno lub najkrótsze słowa są takie same.
 $NWP(\cdot, 2, 1) = n - 1$, kiedy najkrótsze słowa się różnią, ponieważ najkrótsze słowa mogą się różnić pierwszym znakiem 
 ciągu.

\end{section}

% \begin{section}
% {$NWP(\cdot, 2, \cdot)$ $\in$ $NPC$}
% \begin{subsection}
%   {Problem pokrycia wierzchołkowego (vertex cover)}
%   Problem pokrycia wierzchołkowego jest NP-zupełny \cite{janczewski}.
%   Dany jest graf $G = (V, E)$ oraz liczba $k \in \mathbb{N}$. Czy istnieje podzbiór $V' \subseteq V$ o 
%   mocy $k$ taki, że każda krawędź ma przynajmniej jeden koniec w $V'$?
% \end{subsection}
% \begin{subsection}
%   {Redukcja z problemu pokrycia wierzchołkowego do problemu $NWP(\cdot, 2, \cdot)$}
%   Taka redukcja jest możliwa w realizacji i udowodniona w dokumencie \cite{maier}.
% \end{subsection}
% \end{section}

\printbibliography

\end{document}